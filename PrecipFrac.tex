\NeedsTeXFormat{LaTeX2e}[2011/06/27]
\documentclass[12pt]{article}[2007/10/19]

\usepackage{cool}[2006/12/29]

\usepackage[utf8]{inputenc}[2008/03/30]
\usepackage[T1]{fontenc}[2005/09/27]
\usepackage[letterpaper,margin=1in]{geometry}[2010/09/12]
\usepackage{indentfirst}[1995/11/23]

\usepackage{enumitem}[2011/09/28]

\usepackage{amsmath}[2013/01/14]
\usepackage{amssymb}[2013/01/14]
\usepackage{textcomp}[2017/04/05]

\usepackage{graphicx}[1999/02/16]

\usepackage[squaren]{SIunits}[2007/12/02]

\usepackage{listings}[2015/06/04]
\usepackage[scaled=0.85]{beramono}[2004/01/31]

\usepackage{xcolor}[2016/05/11]

\usepackage[backend=biber,
  citestyle=authoryear,
  maxbibnames=5]{biblatex}[2014/06/25]
\addbibresource{PrecipFrac.bib}

\renewcommand\rmdefault{put}
\renewcommand\familydefault\rmdefault

\frenchspacing

\title{Precipitation Fraction}
\author{Sean Patrick Santos}

\DeclareMathOperator\rank{rank}

% Don't use ``ln'' for log base e, and show parentheses.
\Style{LogBaseESymb=log,LogParen=p}

% Default to monospace code, in fortran.
\lstset{language=[08]Fortran,basicstyle=\ttfamily}

\begin{document}

\maketitle

\section{Precipitation Fraction Basics}

Consider a partition of a grid cell $k$ into two parts, a high-precipitation
region $R_{P,k}$, and a low-precipitation area $R_{N,k}$. Consider also that the
grid cell is divided into a high-cloud region $R_{C,k}$, and a low-cloud region
$R_{E,k}$. $R_{P,k}$ is chosen to be the region with the highest concentration
of precipitation, containing a fraction $r_p$ of the precipitation mass, while
$1-r_p$ is in $R_{N,k}$. Similarly, the cloud fraction $R_{C,k}$ contains a
fraction $r_c$ of cloud mass, while $1-r_c$ is in $R_{E,k}$. The fractional area
of $R_{P,k}$ and $R_{C,k}$ are $f_p$ and $f_c$, respectively. In the total
absence of precipitation or cloud mass, these regions are undefined, but it is
convenient to consider $f_p$ to be zero if there is no precipitation, and
similarly for $f_c$.

Now consider a grid cell which initially has no precipitation, but does have
cloud. Within the grid cell, there will be a cloud mass mixing ratio in the
vicinity of any given point, $q_c$, and a rate of change of cloud mass mixing
ratio, $\dot{q}_c$. The probability density function of these variables within
the grid cell can be labeled as $D(q_c, \dot{q}_c)$. The time evolution of
this function is then:

\begin{align}
  \frac{d}{d t} D(q_c, \dot{q}_c) &= \dot{q}_c \frac{\partial}{\partial q_c}
  D(q_c, \dot{q}_c) + \ddot{q}_c \frac{\partial}{\partial \dot{q}_c} D(q_c,
  \dot{q_c})
\end{align}

We can also look at the PDF of $q_c$ alone, which we can label $D_c$:

\begin{align}
  D_c(q_c) &= \int_{-\infty}^\infty D(q_c, \dot{q}_c) d\dot{q}_c \\ \frac{d
    D_c(q_c)}{dt} &= \int_{-\infty}^\infty \dot{q}_c \frac{\partial}{\partial q_c} D(q_c, \dot{q}_c)
  d\dot{q}_c
\end{align}

A variable of interest is the mean mass mixing ratio, $\bar{q}_c$:

\begin{align}
  \bar{q}_c &= \int_0^\infty q_c D_c(q_c) dq_c \\
  \frac{d \bar{q}_c}{dt} &= 
\end{align}

\end{document}
